% This is a simple sample document.  For more complicated documents take a look in the exercise tab. Note that everything that comes after a % symbol is treated as comment and ignored when the code is compiled.

\documentclass[a4paper]{article} % \documentclass{} is the first command in any LaTeX code.  It is used to define what kind of document you are creating such as an article or a book, and begins the document preamble

\usepackage{amsmath} % \usepackage is a command that allows you to add functionality to your LaTeX code
\usepackage[lmargin=3.0cm,rmargin=2.0cm,tmargin=2.0cm,bmargin=2.0cm]{geometry} % show page margins without modifying the page layout

\title{Spectre Monotile} % Sets article title
\author{Iris Kohlbecker} % Sets authors name
\date{Sommerprojekt 2023} % Sets date for date compiled

% The preamble ends with the command \begin{document}
\begin{document} % All begin commands must be paired with an end command somewhere
    \maketitle % creates title using information in preamble (title, author, date)
    
    \section{Einführung} % creates a section

    Dieser Bericht dient als Rezept für die Dokumentation eines Projekts
    in Text und Video.
    In diesem Bericht dokumentiere ich als Beispielprojekt den Einstein ``Spectre''.
    Ein Einstein ist eine Kachel, mit der man die zweidimensionale Ebene bis ins 
    Unendliche auslegen kann, ohne dass sich das Muster je wiederholt.
    Der Einstein ``Spectre" ist die erste solche Kachel und wurde erst im Mai 2023
    von Mathematikern entdeckt.

    Der Bericht zeigt ausserdem, wie man alle Unterlagen eines Projekts
    in einer Source-Verwaltung ablegt.
    Und gewisse Tastenkombinationen zur Vereinfachung der Dokumentation.
    
    \textbf{Hello World!} Today I am learning \LaTeX. %notice how the command will end at the first non-alphabet charecter such as the . after \LaTeX
     \LaTeX{} is a great program for writing math. I can write in line math such as $a^2+b^2=c^2$ %$ tells LaTexX to compile as math
     . I can also give equations their own space: 
    \begin{equation} % Creates an equation environment and is compiled as math
    \gamma^2+\theta^2=\omega^3
    \end{equation}
    If I do not leave any blank lines \LaTeX{} will continue  this text without making it into a new paragraph.  Notice how there was no indentation in the text after equation (1).  
    Also notice how even though I hit enter after that sentence and here $\downarrow$
     \LaTeX{} formats the sentence without any break.  Also   look  how      it   doesn't     matter          how    many  spaces     I put     between       my    words.
    
    For a new paragraph I can leave a blank space in my code.

    \section{Werkzeuge} % creates a section

    In diesem Projekt benutze ich folgende Programme:
    \begin{itemize}
        \item \textsl{VisualStudio Code} (v1.80.2)
        \item VSCode Erweiterung \textsl{LaTeX Workshop} (v9.13.3)
        \item LaTeX Distribution \textsl{MikTeX} (v23.4), benötigt von \textsl{LaTeX Workshop}
        \item \textsl{Perl} Distribution strawberry-perl (v5.32.1.1), benötigt von \textsl{MikTeX}
        \item \textsl{KeePass2} (v2.54), benötigt zum Abspeichern von Passwörtern
        \item \textsl{GitHub}, https://github.com/, als öffentliche Ablage und Versionsverwaltung
        \item \textsl{TortoiseGit} (v2.14.0), benötigt für git
        \item \textsl{git} (v2.41.0), benötigt von \textsl{TortoiseGit}
    \end{itemize}

    \section{Pfad-Variable} % creates a section
    Windows kennt eine globale Umgebungsvariable mit dem Namen ``Path''.
    In dieser Variable ist eine Liste von Verzeichnispfaden gespeichert.
    Wenn ein Programm ein anderes Programm aufruft, 
    sucht Windows das Programm in diesen Verzeichnissen.
    Nur wenn es in einem dieser Verzeichnisse gefunden wird,
    kann es ausgeführt werden.
    Sonst gibt es eine Fehlermeldung.

    Genau genommen gibt es zwei Umgebungsvariablen mit dem Namen ``Path'':
    eine für den eigenen Benutzer und eine für das ganze System,
    das heisst für alle Benutzer.

    Um der Pfadvariable des eigenen Benutzers einen weiteren Verzeichnispfad hinzuzufügen:
    \begin{quote}
        Start $\triangleright$ Einstellungen $\triangleright$
        Systemeigenschaften $\triangleright$ Umgebungsvariablen
        $\triangleright$ Benutzervariablen für [User] $\triangleright$
        Path $\triangleright$ Bearbeiten
    \end{quote}

    Zum Beispiel hat der Installer von \textsl{MikTeX} die Pfadvariable
    nicht automatisch erweitert.
    Ich habe von Hand den Pfad für \textsl{MikTeX} eingetragen:
    \begin{quote}
        C:\textbackslash Users\textbackslash Iris\textbackslash AppData\textbackslash Local\textbackslash Programs\textbackslash MiKTeX\textbackslash miktex\textbackslash bin\textbackslash x64\textbackslash 
    \end{quote}

    \section{VSCode Einstellungen für LaTeX Workshop} % creates a section
   
    Output-Verzeichnis für \textsl{MikTeX} setzen:
    \begin{quote}
        VScode $\triangleright$ Manage $\triangleright$ Settings $\triangleright$ extentions $\triangleright$ LaTex $\triangleright$ Latex-workshop $\triangleright$ Latex: Out Dir
        geändert auf "\%DIR\%/../Output/LaTeX"
    \end{quote}

    \section{Hilfeseiten} % creates a section
  
    \begin{itemize}
        \item StackExchange für TeX: https://tex.stackexchange.com/
        \item StackOverflow: https://stackoverflow.com/,
        beim Suchen das Thema in eckige Klammern setzen,
        z.B.: [python]...
    \end{itemize}

    \section{Tastenkombinationen für VSCode} % creates a section
  
    \begin{itemize}
       \item Zeige Suchfenster für alle Befehle: Ctrl + Shift + P
       \item Ganze Zeile löschen: Ctrl + Shift + K
       \item für weitere Kombinationen: Google-Suche: VSCode cheat sheet
    \end{itemize}
      
    \section{GitHub} % creates a section
  
    Ich habe einen User bei GitHub angelegt 
    und habe die Begrüssungsseite ``README.md'' erstellt.
    Statt zu speichern spricht man bei GitHub von einem Commit.
    Mit jedem Commit wird eine Datei nicht überschrieben,
    sondern eine neue Version davon gespeichert.
    Das erlaubt, frühere Versionen von jeder Datei anzuschauen.
    
    Ich habe mein git repository auf GitHub.com gepusht.
    Dies führte ich mit hilfe von TortoiseGit durch.
    Um die Geschichte anzusehen, füge ich der URL (in der Username und Projektname schon 
    enthalten sind) ``/commits/master'' hinzu.
    
\end{document} % This is the end of the document